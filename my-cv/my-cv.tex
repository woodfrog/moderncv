%% Inicio del archivo `template-es.tex'.
%% Copyright 2006-2013 Xavier Danaux (xdanaux@gmail.com).
%
% This work may be distributed and/or modified under the
% conditions of the LaTeX Project Public License version 1.3c,
% available at http://www.latex-project.org/lppl/.


\documentclass[11pt,a4paper,sans]{moderncv}   % opciones posibles incluyen tamaño de fuente ('10pt', '11pt' and '12pt'), tamaño de papel ('a4paper', 'letterpaper', 'a5paper', 'legalpaper', 'executivepaper' y 'landscape') y familia de fuentes ('sans' y 'roman')

% temas de moderncv
\moderncvstyle{classic}                        % las opciones de estilo son 'casual' (por omision),'classic', 'oldstyle' y 'banking'
\moderncvcolor{purple}                          % opciones de color 'blue' (por omision), 'orange', 'green', 'red', 'purple', 'grey' y 'black'
%\renewcommand{\familydefault}{\sfdefault}    % para seleccionar la fuente por omision, use '\sfdefault' para la fuente sans serif, '\rmdefault' para la fuente roman, o cualquier nombre de fuente
%\nopagenumbers{}                             % elimine el comentario para suprimir la numeracion automatica de las paginas para CVs mayores a una pagina

% codificacion de caracteres
%\usepackage[utf8]{inputenc}                  % reemplace con su codificacion
%\usepackage{CJKutf8}                         % si necesita usa CJK para redactar su CV en chino, japones o coreano

% ajustes para los margenes de pagina
\usepackage[scale=0.75]{geometry}
%\setlength{\hintscolumnwidth}{3cm}           % si desea cambiar el ando de la columna para las fechas

% datos personales
\name{Jiacheng(Jaycee)}{Chen}
% \title{T\'itulo del CV (opcional)}                   % dato opcional, elimine la linea si no desea el dato
%\address{8888 University Drive}{Burnaby, B.C. Canada} % dato opcional, elimine la linea si no desea el dato
\phone[mobile]{+1~(778)~885~7227}                     % dato opcional, elimine la linea si no desea el dato
% \phone[fixed]{+2~(345)~678~901}                      % dato opcional, elimine la linea si no desea el dato
% \phone[fax]{+3~(456)~789~012}                       % dato opcional, elimine la linea si no desea el dato
\email{jca348@sfu.ca}                                 % dato opcional, elimine la linea si no desea el dato
\homepage{jcchen.me}                           % dato opcional, elimine la linea si no desea el dato
%\extrainfo{informacion adicional}                    % dato opcional, elimine la linea si no desea el dato
%\photo[64pt][0.4pt]{picture}                         % '64pt' es la altura a la que la imagen debe ser ajustada, 0.4pt es grosor del marco que lo contiene (eliga 0pt para eliminar el marco) y 'picture' es el nombre del archivo; dato opcional, elimine la linea si no desea el dato
%\quote{Alguna cita (opcional)}                       % dato opcional, elimine la linea si no desea el dato

% para mostrar etiquetas numericas en la bibliografia (por omision no se muestran etiquetas), solo es util si desea incluir citas en en CV
%\makeatletter
%\renewcommand*{\bibliographyitemlabel}{\@biblabel{\arabic{enumiv}}}
%\makeatother

% bibliografia con varias fuentes
%\usepackage{multibib}
%\newcites{book,misc}{{Libros},{Otros}}
%----------------------------------------------------------------------------------
%            contenido
%----------------------------------------------------------------------------------
\begin{document}
%\begin{CJK*}{UTF8}{gbsn}                     % para redactar el CV en chino usando CJK
\maketitle

\section{Overview}
\cvitem{-}{Solid programming skills in \textbf{C/C++}, \textbf{Python} with rich project experience}
\cvitem{-}{Deep interest in computer vision and distributed computing, familiar with deep learning frameworks \textbf{Tensorflow}, \textbf{Pytorch} and distributed frameworks \textbf{Hadoop}, \textbf{Spark}}
\cvitem{-}{Rich experience with common tools including \textbf{Git}, \textbf{Latex}, \textbf{Matlab}, etc.}

\section{Education}
\cventry{2016-Present}{Simon Fraser University}{Bunarby}{BC}{Canada}
	{B.Sc in Computing Science, Dual Degree Program, GPA: 4.17/4.33}  % Los argumentos del 3 al 6 pueden permanecer vacios
\cventry{2014-Present}{Zhejiang University}{Hangzhou}{Zhejiang}{China}
	{B.Sc in Computer Science and Technology, GPA: 3.93/4.0}

%\section{Education}
%\cvitem{}{\emph{T\'itulo}}
%\cvitem{sinodares}{Sinodales}
%\cvitem{descripci\'on}{Una breve descripci\'on de la tesis}

\section{Research Experience}
\cventry{May 2017-}{Research Assistant}
{\newline VML Lab}{Simon Fraser University}{Advisor: Prof.Greg Mori}
{Research in computer vision and deep learning%
\begin{itemize}%
\item Researched on the analysis and prediction of human activity, and also on generative models for generating controllable images
\item Designed and implemented a framework for multi-person future forecasting and applied it on complex sports forecasting
%  \begin{itemize}%
%  \item Sub-logro (a);
%  \item Sub-logro (b), con sub-sub-logros (¡evite hacer esto!);
%    \begin{itemize}
%    \item Sub-sub-logro i;
%    \item Sub-sub-logro ii;
%    \item Sub-sub-logro iii;
%    \end{itemize}
%  \item Sub-logro (c);
%  \end{itemize}
\item Extracted pose sequences for \textit{Volleyball Dataset}, which is a common dataset for group activity recognition.
\end{itemize}}

\cventry{Sept 2017-}{Research Assistant}{\newline Big Data Research Project}{Simon Fraser University}{Advisor: Prof.Ryan Shea}
{Research in distributed computing system integrated with computer vision%
\begin{itemize}%
\item Built up a distributed computing system for large-scale video processing based on Spark Streaming, FFMPEG, OpenCV
\item Analyzed the efficiency and energy performance of proposed system on SFU Cloud?s 8-node cluster
\item Deployed our scalable system on SFU Cloud cluster in Burnaby campus for real-time vehicle monitoring
\end{itemize}
}


\section{Publication and Manuscript}
\cventry{Dec 2017}{ Learning to Forecast Videos of Human Activity with Multi-granularity Models and Adaptive
Rendering}
{}
{\newline Mengyao Zhai, \underline{Jiacheng Chen}, Ruizhi Deng, Ligeng Zhu, Lei Chen and Greg Mori}
{\newline ArXiv Preprint}
{A hierarchical framework for forecasting complex human videos}

\section{Honours and Awards}
\cventry{2017}{Meritorious Prize}
{Mathematical Contest in Modeling(MCM)}
{}
{}
{
\begin{itemize}
\item Top 7\% in all participants of the competition
\item Implemented a Cellular Automata for simulating and analyzing highway traffic flow
\end{itemize}}

\cventry{2017}{First Class Entrance Scholarship}
{Simon Fraser University}
{}
{}
{The scholarship rewards top 10\% students in SFU-ZJU Dual Degree Program}

\cventry{2016}{First Prize Academic Scholarship}
{Zhejiang University}
{}
{}
{The scholarship rewards the top 5\% student according to academic behavior}

\section{Selected Projects}

\cventry{April 2017}{Action Recognition Exploration}{\textit{Github link}}{}{}
{%
\begin{itemize}%
\item Implemented an intelligent deep-learning-based human action recognizer which can be accessed from browser to recognize the videos from local files
%
\item Used Tensorflow to build a two-stream neural network for high-accuracy recognition as part of the backend of the app
\end{itemize}
}

\cventry{Dec 2016}{Intelligent Vegetable Classifier}{}{}{}
{%
\begin{itemize}%
\item Created an intelligent classifier based on state-of-art convolutional neural network to identify among 50 different kinds of fruits and vegetables with over 50\% top-1 accuracy%
\item Referred to several related papers on color constancy and applied a special preprocessing technique to enhance the model?s stability under different light environments
\item Implemented a web application for the classifier with Django to make it both accessible for desktop and mobile users
\end{itemize}
}

\cventry{Oct 2016}{Basic Shell}{\textit{Github link}}{}{}
{%
\begin{itemize}%
\item Created a shell with C and system calls to simulate the performance of bash%
\item Implemented pipe using inter-process communication to make the shell support more complex and integrated commands
\item Achieved functionality of job control by dealing with internal signals and enabled users to manage their tasks easily and efficiently
\end{itemize}
}


\cventry{Sept 2016}{SFU Wechat Assistant}{\textit{Github link}}{}{}
{%
\begin{itemize}%
\item Built an Wechat assistant cooperated with another developer for reporting SFU calendar automatically by sending notifications about courses and events%
\item Deployed the assistant on our VPS so that it can be used by anyone who subscribes our corresponding public Wechat account
\end{itemize}
}

\cventry{June 2016}{MiniSQL}{\textit{Github link}}{}{}
{%
\begin{itemize}%
\item Designed a mini database system using Python with a group of three and successfully passed MySQL-based test cases
%
\item Applied unittest on core modules with automatic testing tools to guarantee high quality of the code
%
\item Implemented a SQL interpreter with PLY and Backus Normal Form to parse SQL language
%
\item Implemented record manager with efficient data structures to manipulate massive conversion between database records and binary data efficiently
\end{itemize}
}

\cventry{Feb 2016}{FPGA Greedy Snake Game}{\textit{Github link}}{}{}
{%
\begin{itemize}%
\item Implemented the classic snake game on FPGA with Verilog HDL and enabled players to play it on any device with VGA port%
\item Created different patterns by plotting bitmaps to prettify the game with the theme of Pac-Man
\item Designed algorithm based on geometrical principles to keep the shape of snake while moving and rotating
\end{itemize}
}








%\cvitem{hobby 1}{Descripci\'on}
%\cvitem{hobby 2}{Descripci\'on}
%\cvitem{hobby 3}{Descripci\'on}

\renewcommand{\listitemsymbol}{-~}            % para cambiar el simbolo para las listas

% Las publicaciones tomadas de un archivo de BibTeX sin usar multibib\renewcommand*{\bibliographyitemlabel}{\@biblabel{\arabic{enumiv}}}

\nocite{*}
\bibliographystyle{plain}
\bibliography{publications}                   % 'publications' es el nombre del archivo BibTeX

% Las publicaciones tomadas de un archivo BibTeX usando el paquete multibib
%\section{Publicaciones}
%\nocitebook{book1,book2}
%\bibliographystylebook{plain}
%\bibliographybook{publications}              % 'publications' es el nombre del archivo BibTeX
%\nocitemisc{misc1,misc2,misc3}
%\bibliographystylemisc{plain}
%\bibliographymisc{publications}              % 'publications' es el nombre del archivo BibTeX

%\clearpage\end{CJK*}                          % si esta redactando su CV en chino usando CJK, \clearpage es requerido por fancyhdr para que funcione correctamente con CJK, aunque esto eliminara la numeracion de pagina al dejar \lastpage como no definido
\end{document}


%% fin del archivo `template-es.tex'.
